% !TEX root = Projektdokumentation.tex

% Hinweis: der Titel muss zum Inhalt des Projekts passen und den zentralen Inhalt des Projekts deutlich herausstellen

\newcommand{\IHKlogo}{logo_handelskammer.png}
\newcommand{\titelOne}{Vereinheitlichung und Digitalisierung}
\newcommand{\titelTwo}{der Materialeingangsprüfungsberichte}
\newcommand{\untertitelOne}{mithilfe einer zentralen Datenbankanbindung}
\newcommand{\untertitelTwo}{im Synchronbereich}
\newcommand{\kompletterTitel}{\titelOne{}\\\titelTwo{}\\\untertitelOne{}\\\untertitelTwo{}}

\newcommand{\amqp}{amqp.png}
\newcommand{\netzplan}{netzplan.png}

\newcommand{\autorName}{Rico Krüger}

\newcommand{\betriebLogo}{logo-bs_sl.png}
\newcommand{\betriebLogoLower}{logo-bs_sl.png}
\newcommand{\betriebName}{Berliner Synchron GmbH}

\newcommand{\betriebAnschrift}{EUREF-Campus 10-11}
\newcommand{\betriebOrt}{10829 Berlin}

\newcommand{\ausbildungsberuf}{Fachinformatiker für Anwendungsentwicklung}
\newcommand{\betreff}{Dokumentation zur betrieblichen Projektarbeit}

\newcommand{\abgabeOrt}{Berlin}
\newcommand{\abgabeTermin}{15.06.2018}