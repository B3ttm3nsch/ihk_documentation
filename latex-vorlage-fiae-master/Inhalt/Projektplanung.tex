% !TEX root = ../Projektdokumentation.tex
\section{Projektplanung} 
\label{sec:Projektplanung}
Der Projektplanung ging ein Meeting zwischen der Geschäftsführung, einer Mitarbeiterin, welche zum Großteil Materialeingangsberichte erfasst und der IT-Abteilung voraus. Nachdem die Geschäftsführung sich nach einer Zusammenfassung der Materialeingangsprüfungsberichte für einen bestimmten Kunden erkundigte bzw. nach der Möglichkeit einer solchen, wurde schnell klar, dass dies auf Basis der aktuellen Arbeitsweise nicht gewährleistet werden kann. Für eine übersichtliche, gefilterte Darstellung solcher Berichte würde eine gewisse Standardisierung und zentrale Speicherung der Daten benötigt. Auf dieser Grundlage entstand die Idee eine Erweiterung für Copper zu entwickeln.

\subsection{Projektphasen}
\label{sec:Projektphasen}
% In welchem Zeitraum und unter welchen Rahmenbedingungen (\zB Tagesarbeitszeit) findet das Projekt statt?
Aufgrund des von der \ac{IHK} vorgegebenen begrenzten Zeitrahmens, wurde das Projekt in ein Teilprojekt gewandelt. Nach der positiven Rückmeldung auf meinen Projektantrag wurde das Projekt in den Tagesablauf meiner betrieblichen Arbeit integriert und mit ca. 30 Wochenstunden bearbeitet.

\subsection{Zeitplanung}
\label{sec:Zeitplanung}
% Verfeinerung der Zeitplanung, die bereits im Projektantrag vorgestellt wurde.
Für die Umsetzung des Projektes stehen seitens der Anforderungen der \ac{IHK} 70 Stunden zur Verfügung. Diese wurden zur Antragstellung auf die einzelnen Phasen verteilt. Die grobe Zeitplanung der Hauptphasen kann nachfolgender Tabelle~\ref{tab:Zeitplanung} Zeitplanung entnommen werden. Eine ausführlichere Zeitplanung findet sich im~\Anhang{subsec:Zeitplan}.

\tabelle{Zeitplanung}{tab:Zeitplanung}{ZeitplanungKurz}

\subsection{Abweichungen vom Projektantrag}
\label{sec:AbweichungenProjektantrag}
Bei der Erstellung des Projektantrags wurde geplant, das Projekt auf einer bereits vorhandenen Testumgebung aufzusetzen und in Absprache mit dem Entwickler von Copper anzufertigen. Da der Entwickler von Copper während der Projektarbeit weder erreichbar, noch ein Zugriff auf die Testumgebung vorhanden war, musste das Projekt mit den zur Verfügung stehenden Daten zur Zeit der Erstellung des Projektantrags nachgebaut werden. Somit ergaben sich erhebliche Änderungen im Vergleich zum Projektantrag. Ferner konnte aufgrund der am Ende fehlenden Zeit durch die vorhergehenden, nicht eingeplanten Entwicklungsarbeiten keine automatisierten Tests des entwickelnden Materialeingangsberichts mehr durchgeführt werden. Das hätte das zusätzliche Aufsetzen einer Testumgebung bedurft, welche am Ende nicht einmal den tatsächlichen Produktionseinsatz der Anwendung sichergestellt hätte. Daher wurde in Absprache mit dem Ausbilder beschlossen, lediglich ein Testprotokoll zu erstellen. Weiterhin wurde auf die Erstellung einer Entwicklerdokumentation verzichtet.

\subsection{Ressourcenplanung}
\label{sec:Ressourcenplanung}
% Detaillierte Planung der benötigten Ressourcen (Hard-/Software, Räumlichkeiten \usw).
% \Ggfs sind auch personelle Ressourcen einzuplanen (\zB unterstützende Mitarbeiter).
% Hinweis: Häufig werden hier Ressourcen vergessen, die als selbstverständlich angesehen werden (\zB PC, Büro).
Alle benötigten Ressourcen zur Durchführung des Projekts sind bereits im Unternehmen vorhanden bzw. können online heruntergeladen werden. Eine Auflistung befindet sich im ~\Anhang{app:Ressourcen}.

\subsection{Entwicklungsprozess}
\label{sec:Entwicklungsprozess}
% Welcher Entwicklungsprozess wird bei der Bearbeitung des Projekts verfolgt (\zB Wasserfall, agiler Prozess)?
Die Entwicklung des Projekts  erfolgt nach Rücksprache mit der verantwortlichen Mitarbeiterin für Qualitätsprüfung im Unternehmen. Ansonsten wird das Projekt weitestgehend nach dem Wasserfallmodell in Eigenregie erstellt.
