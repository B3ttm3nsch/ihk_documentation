% !TEX root = ../Projektdokumentation.tex
\section{Dokumentation}
\label{sec:Dokumentation}
% Wie wurde die Anwendung für die Benutzer/Administratoren/Entwickler dokumentiert (\zB Benutzerhandbuch, API-Dokumentation)?
Diese Projektdokumentation wurde im Rahmen der IHK - Abschlussprüfung zum IT-Fachinformatiker für Anwendungsentwicklung mit \LaTeX{ } erstellt.

% Hinweis: Je nach Zielgruppe gelten bestimmte Anforderungen für die Dokumentation (\zB keine IT-Fachbegriffe in einer Anwenderdokumentation verwenden, aber auf jeden Fall in einer Dokumentation für den IT-Bereich).

\subsection{Benutzerdokumentation}
\label{subsec:Benutzerdokumentation}
%Die Entwicklerdokumentation wurde mittels PHPDoc\footnote{Vgl. \cite{phpDoc}} automatisch generiert. Ein beispielhafter Auszug aus der Dokumentation einer Klasse findet sich im \Anhang{app:Doc}.
Eine vorläufige Benutzerdokumentation mit entsprechenden Hinweisen für die Bedienung wurde mit Microsoft-Word geschrieben und in einen für Handbücher vorgesehenen Netzwerkordner im PDF-Format gespeichert. Ein Auszug befindet sich im ~\Anhang{app:Benutzerdokumentation}.

\Zwischenstand{Dokumentation}{Dokumentation}
