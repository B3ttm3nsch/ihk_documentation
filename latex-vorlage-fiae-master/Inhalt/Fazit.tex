% !TEX root = ../Projektdokumentation.tex
\section{Fazit} 
\label{sec:Fazit}

\subsection{Soll-/Ist-Vergleich}
\label{sec:SollIstVergleich}
% Wurde das Projektziel erreicht und wenn nein, warum nicht?
% Ist der Auftraggeber mit dem Projektergebnis zufrieden und wenn nein, warum nicht?
% Wurde die Projektplanung (Zeit, Kosten, Personal, Sachmittel) eingehalten oder haben sich Abweichungen ergeben und wenn ja, warum?
% Hinweis: Die Projektplanung muss nicht strikt eingehalten werden. Vielmehr sind Abweichungen sogar als normal anzusehen. Sie müssen nur vernünftig begründet werden (\zB durch Änderungen an den Anforderungen, unter-/überschätzter Aufwand).
%\paragraph{Beispiel (verkürzt)}
% Tabelle~\ref{tab:Vergleich} Zeitplanung
Auch wenn ich mir den Projektablauf und das Ergebnis zum Zeitpunkt der Antragsstellung gänzlich anders vorgestellt habe, sind die Kollegen mit dem neuen Formular sehr zufrieden und würden es gerne schon in Betrieb nehmen.\\
Insgesamt weicht die geplante Zeit von der tatsächlich benötigten Zeit nur geringfügig ab. Betrachtet man jedoch die einzelnen Teile, fällt auf, dass es nicht in Gänze einzuschätzen war, wie groß der tatsächliche Aufwand dieser jeweils werden würde. Tabelle~\ref{tab:Vergleich} zeigt einen Vergleich der veranschlagten sowie tatsächlich aufgewendeten Zeit.

\tabelle{Soll-/Ist-Vergleich}{tab:Vergleich}{Zeitnachher.tex}

\subsection{Gewonnene Erkenntnisse}
\label{sec:LessonsLearned}
% Was hat der Prüfling bei der Durchführung des Projekts gelernt (\zB Zeitplanung, Vorteile der eingesetzten Frameworks, Änderungen der Anforderungen)?
Das Projekt in kompletter Eigenregie zu erstellen, kostete zwar einerseits viel Kraft, aber hat mich persönlich weitergebracht. Jeder Schritt in der Erstellung des Projekts machte aus mir einen besseren Entwickler, auch wenn es mir vor Augen führte wieviel Sachen es noch über \acs{Ruby}, \acs{Rails} und die komplette Webprogrammierung es noch zu lernen und zu verinnerlichen gilt, da dies wirklich, ausgenommen der Projektdokumentation, eine Sache ist, die mir Freude bereitet und für die ich gerne morgens zur Arbeit gehe.

\subsection{Ausblick}
\label{sec:Ausblick}
%  Wie wird sich das Projekt in Zukunft weiterentwickeln (\zB geplante Erweiterungen)?
Wenn man für die Anwendung noch ein paar Sicherheitsrichtlinien hinzufügt, die Erstellung von Projekten und Teilprojekten auf ähnlicher Weise wie den Audiobericht generiert, noch ein wenig mehr Zeit dem Testen widmet, könnte diese Anwendung tatsächlich in den Betrieb gehen. Da dies jedoch einen zusätzlichen Mehraufwand für die Firma bedeutet, muss dies noch von der Geschäftsführung entschieden werden. Dabei spielt vor allem der tatsächliche Zeitpunkt der Implementierung von Copper eine Rolle.
