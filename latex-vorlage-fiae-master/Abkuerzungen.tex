% !TEX root = Projektdokumentation.tex

% Es werden nur die Abkürzungen aufgelistet, die mit \ac definiert und auch benutzt wurden. 
%
% \acro{VERSIS}{Versicherungsinformationssystem\acroextra{ (Bestandsführungssystem)}}
% Ergibt in der Liste: VERSIS Versicherungsinformationssystem (Bestandsführungssystem)
% Im Text aber: \ac{VERSIS} -> Versicherungsinformationssystem (VERSIS)

% Hinweis: allgemein bekannte Abkürzungen wie z.B. bzw. u.a. müssen nicht ins Abkürzungsverzeichnis aufgenommen werden
% Hinweis: allgemein bekannte IT-Begriffe wie Datenbank oder Programmiersprache müssen nicht erläutert werden,
%          aber ggfs. Fachbegriffe aus der Domäne des Prüflings (z.B. Versicherung)

% Die Option (in den eckigen Klammern) enthält das längste Label oder
% einen Platzhalter der die Breite der linken Spalte bestimmt.
\begin{acronym}[WWWWWW]
	\acro{IHK}{Industrie- und Handelskammer}
	\acro{BSG}{Berliner Synchron GmbH}
	\acro{MEP}{Materialeingangsprüfungsbericht}
	\acro{QC}{Quality Check}
	\acro{Ruby}{Objektorientierte Programmiersprache}
	\acro{Framework}{Programmiergerüst}
	\acro{Rails}{\acs{Framework} für Webanwendungen}
	\acro{RoR}{\acs{Ruby} on \acs{Rails}}
	\acro{SQL}{Structured Query Language: Programmiersprache für Datenbankabfrage}
	\acro{MySQL}{My Structured Query Language: Relationale Datenbank}
	\acro{HTTP}{Hypertext Transfer Protocol: Protokoll zur Datenübertragung}
	\acro{mysql2}{\acs{MySQL}-Datenbank-Connector}
	\acro{gem}{Bezeichnung für Bibliotheken und Frameworks in Ruby}
	\acro{HTML}{Hypertext Markup Language: Programmiersprache zur Strukturierung einer Webseite}
	\acro{HAML}{\acs{HTML} abstraction markup language: Programmiersprache zur Generierung von HTML-Code; Erlaubt u.a.\ die Ausführung von Ruby-Code}
	\acro{CSS}{Cascading Style Sheets: Programmiersprache zum Designen von Webseiten}
	\acro{Javascript}{Scriptsprache zum Erstellen dynamischer Webseiten}
	\acro{Bootstrap}{\acs{HTML}-, \acs{CSS}-, \acs{Javascript}-\acs{Framework}}
	\acro{Model}{Datenmodel}
	\acro{View}{Darstellung}
	\acro{Controller}{Programmsteuerung}
    \acro{MVC}{\acs{Model}-\acs{View}-\acs{Controller}: Architekturdesign zur Trennung der Software in drei Komponenten: \acs{Model}, \acs{View} und \acs{Controller}}
    \acro{CRUD}{Create Read Update Delete}
    \acro{GUI}{Graphical User Interface}
    \acro{rvm}{Ruby Version Manager}
    \acro{IDE}{Integrated Development Envrionment}
    
	

\end{acronym}
